@tex
%% Cygnus revisions to texinfo.tex, TeX macros to handle texinfo files
%   Copyright (C) 1991 Free Software Foundation, Inc.
%% Maintained at Cygnus Support as: 
%%$Id: texiplus.tex,v 1.1.1.1 1998/03/23 04:42:12 law Exp $
  
%This texi+.tex file is free software; you can redistribute it and/or
%modify it under the terms of the GNU General Public License as
%published by the Free Software Foundation; either version 1, or (at

{\let\fsfvn=\texinfoversion
\xdef\texinfoversion{\fsfvn\ (Cygnus)}}
\message{Loading Cygnus texinfo revisions [\texinfoversion]:}
  
% Print the version number if in a .fmt file.
\everyjob{\message{[Cygnus Texinfo \texinfoversion]}}

\globaldefs=1 % Escape the bounds of @tex/@end tex surrounding us

%%%%%%%%%%%%%%%%%%%%%%%%%%%%%%%CROPMARKS%%%%%%%%%%%%%%%%%%%%%%%%%%%%%%
% These differ only slightly from FSF defaults; all crop dimens are
% defined here (whether different or not) for ease in revising.
\cornerlong=1pc\cornerthick=.3pt	% These set size of cropmarks
\outerhsize=7in
\outervsize=9in
%%%%%%%%%%%%%%%%%%%%%%%%%%%%%%%%%%%%%%%%%%%%%%%%%%%%%%%%%%%%%%%%%%%%%%
		
\newif\ifdraft\drafttrue % Extra markings; turn off with @finalout

% There are only two small changes to standard \title from texinfo 
% (1) to include DRAFT marking on title page unless @finalout
% (2) to include a title-sized \tt font
% However, since \title is local to \titlepage, we have to redefine
% *that* whole damned thing.  
%%%%%%%%%%This is sensitive to conflict w/FSF changes!****************
\def\titlepage{\begingroup \parindent=0pt \textfonts
   \let\subtitlerm=\tenrm
% I deinstalled the following change because \cmr12 is undefined.
% This change was not in the ChangeLog anyway.  --rms.
%   \let\subtitlerm=\cmr12
   \def\subtitlefont{\subtitlerm \normalbaselineskip = 13pt \normalbaselines}%
   %
   \def\authorfont{\authorrm \normalbaselineskip = 16pt \normalbaselines}%
   %
   % Leave some space at the very top of the page.
   \vglue\titlepagetopglue
   %
   % Now you can print the title using @title.
   \def\title{\parsearg\titlezzz}%
%changes from FSF only in following line:
   \def\titlezzz##1{{\let\tentt=\titlett
                    \leftline{\titlefont{##1\ifdraft\hfill DRAFT\fi }}}
%end real changes from FSF version
		    % print a rule at the page bottom also.
		    \finishedtitlepagefalse
%%%%%%%%%%%%%%%%%%%%%%%%%%%%%%%%%%%%%%%%%%%%%%%%%%%%%%%%%%%%%%%%%%%%%%
% The ``width \hsize'' below should be unnecessary, since the rule is
% supposed to default to the smallest enclosing box...but *something*
% in texiplus breaks that default, so we force it to hsize.
		    \vskip4pt \hrule height 4pt width \hsize \vskip4pt}%
%%%%%%%%%%%%%%%%%%%%%%%%%%%%%%%%%%%%%%%%%%%%%%%%%%%%%%%%%%%%%%%%%%%%%%
   % No rule at page bottom unless we print one at the top with @title.
   \finishedtitlepagetrue
   %
   % Now you can put text using @subtitle.
   \def\subtitle{\parsearg\subtitlezzz}%
   \def\subtitlezzz##1{{\subtitlefont \rightline{##1}}}%
   %
   % @author should come last, but may come many times.
   \def\author{\parsearg\authorzzz}%
   \def\authorzzz##1{\ifseenauthor\else\vskip 0pt plus 1filll\seenauthortrue\fi
      {\authorfont \leftline{##1}}}%
   %  
   % Most title ``pages'' are actually two pages long, with space
   % at the top of the second.  We don't want the ragged left on the second.
   \let\oldpage = \page
   \def\page{%
      \iffinishedtitlepage\else
	 \finishtitlepage
      \fi
      \oldpage
      \let\page = \oldpage
      \hbox{}}%
%   \def\page{\oldpage \hbox{}}
}
\def\finishtitlepage{%
% same comment re ``width \hsize'' as on last hrule...
   \vskip4pt \hrule width \hsize height 2pt
   \vskip\titlepagebottomglue
   \finishedtitlepagetrue
}
%%%%%%%%%%end sensitive to conflict w/FSF changes!********************

% Cygnus uses different default page headings and footers
\def\leadline{\ \leaders\hrule height 2.5pt depth -2pt\hfil\ }
\def\manvers{}% Empty default version in case manual doesn't supply
% This is for _explicitly_ turning headers off.  Note that we're more
% literal-minded than the FSF: we don't turn off _footers_!  This
% means that, unlike with the FSF macros, our ``@headings off'' do not
% return to the default Texinfo heading-state of the document.
% Reconsider if this gives trouble...
\def\HEADINGSoff{
\global\evenheadline={\hfil}
\global\evenfootline={\line{\let\,=\thinspace\ftfnt c\,y\,g\,n\,u\,s\quad s\,u\,p\,p\,o\,r\,t\leadline\manvers}}
\global\oddheadline={\hfil}
\global\oddfootline={\line {\let\,=\thinspace\ftfnt c\,y\,g\,n\,u\,s\quad s\,u\,p\,p\,o\,r\,t\leadline\manvers}}}
% SINGLEPAGE:
\def\HEADINGSsingle{
\global\pageno=1\HEADINGSsinglex}
\def\HEADINGSsinglex{
\global\evenfootline={%
{\let\,=\thinspace\ftfnt c\,y\,g\,n\,u\,s\quad s\,u\,p\,p\,o\,r\,t}%
\leadline\manvers}
\global\oddfootline={%
{\let\,=\thinspace\ftfnt c\,y\,g\,n\,u\,s\quad s\,u\,p\,p\,o\,r\,t}%
\leadline\manvers}
\global\evenheadline={\ftfnt\thischapter\leadline
		\ifdraft DRAFT\leadline\fi
		\bf\folio}
\global\oddheadline={\ftfnt\thischapter\leadline
		\ifdraft DRAFT\leadline\fi
		\bf\folio}
}
%
% DOUBLEPAGE:
% use hrule leaders to delimit headings, footings from
%               body; timestamp footer; pagenos outside *bottom* to
%               permit moving chaptername to outside *top* (for easier
%               skimming).  1990 dec 31, pesch@cygnus.com
\def\HEADINGSdouble{
\global\pageno=1\HEADINGSdoublex}
\def\HEADINGSdoublex{
\global\evenfootline={%
{\bf\folio}\ftfnt\leadline\ifdraft DRAFT\quad\fi
\ftfnt\manvers\qquad\today
}
\global\oddfootline={%
{\let\,=\thinspace\ftfnt c\,y\,g\,n\,u\,s\quad s\,u\,p\,p\,o\,r\,t}%
\leadline{\bf\folio}%
}
\global\evenheadline={\ftfnt\thistitle\leadline\ifdraft DRAFT\fi}
\global\oddheadline={\rm\leadline\thischapter}
}

% CHANGE to @finalout ---also use it to remove DRAFT markings from
% title, footers
%% For a final copy, take out the rectangles
%% that mark overfull boxes (in case you have decided
%% that the text looks ok even though it passes the margin).
\def\finalout{\overfullrule=0pt
%% Also take out ``DRAFT'' markings
	\global\draftfalse
			}

%Font overrides; we use PostScript when possible.
\font\textrm=pncr scaled \magstephalf
\font\texttt=pcrr
\font\textbf=pncb scaled \magstephalf 
\font\textit=pncri scaled \magstephalf
\font\textsl=pcrro
\font\textsf=slcrst at 9.5pt
% FIXME: figure out, test dvips smallcaps kluge
%\font\textsc=Times-SmallCaps scaled \magstephalf
\font\ftfnt=phvr at 8pt % Cygnus Support footer-font

\font\defbf=pncb scaled \magstep1 %was 1314

\font\ninett=pcrr at 9pt
\let\indtt=\ninett
\font\indrm=pncr at 9pt
\font\indit=pncri at 9pt
\font\indsl=pcrro at 9pt
\let\indsf=\indrm
\let\indbf=\indrm
\let\indsc=\indrm

\font\chaprm=pncb at 17pt
\font\chapit=pncbi at 17pt
\font\chapsl=pcrbo at 16pt
\font\chaptt=pcrb at 16pt
\font\chapsf=slcrst at 16pt
\let\chapbf=\chaprm
% \chapfonts mod from FSF vn is to include baselineskip, for the few
% cases where a title chapter overflows its line.
\def\chapfonts{\baselineskip=19pt%
  \let\tenrm=\chaprm \let\tenit=\chapit \let\tensl=\chapsl 
  \let\tenbf=\chapbf \let\tentt=\chaptt \let\smallcaps=\chapsc
  \let\tensf=\chapsf \let\teni=\chapi \let\tensy=\chapsy
  \resetmathfonts}


\font\secrm=pncb at 14pt
\font\secit=pncbi at 14pt
\font\secsl=pcrbo at 13pt
\font\sectt=pcrb at 13pt
\font\secsf=slcrst at 13pt
\font\secbf=pncb at 14pt

\font\ssecrm=pncr at 13pt
\font\ssecit=pncri at 13pt
\font\ssecsl=pcrro at 12pt
\font\ssectt=pcrr at 12pt
\font\ssecsf=slcrst at 12pt
\font\ssecbf=pncb at 13pt

\font\titlerm=pncb at 21pt
\font\titlett=pcrr at 20pt
\font\authorrm=pncr scaled \magstep2

\font\truesecrm=pncr at 12pt

% Fonts for short table of contents
\font\shortcontrm=pncr at 12pt 
\font\shortcontbf=pncb at 12pt 
\font\shortcontsl=pncri at 12pt

% \smartitalic{ARG} outputs arg in italics, followed by an italic correction
% unless the following character is such as not to need one.
\def\smartitalicx{\ifx\next,\else\ifx\next-\else\ifx\next.\else\/\fi\fi\fi}
\def\smartitalic#1{{\it #1}\futurelet\next\smartitalicx}
% \smartitalic is used for @emph; FSF keeps trying to make it match
% @var.
%
% \smartslant{ARG} outputs arg in a slanted font, followed by an
% italic correction unless the following character is such as not to need one.
\def\smartslantx{\ifx\next,\else\ifx\next-\else\ifx\next.\else\/\fi\fi\fi}
\def\smartslant#1{{\sl #1}\futurelet\next\smartslantx}

\let\i=\smartitalic
\let\var=\smartslant
\let\dfn=\smartitalic
\let\emph=\smartitalic
\let\cite=\smartitalic

%%%%%%%%%%%%%%%%%%%%%%%%%%%%%%%%%%%%%%%%%%%%%%%%%%%%%%%%%%%%%%%%%%%%%%
% @altref, if called explicitly (inside @iftex), switches the @ref
% printed output so it lacks the square brackets used by default.
% While we're at it, we use double quotes instead of square brackets 
% (or no decorator at all, depending on state of @altref toggle).
% Unfortunately this requires reproducing the entire damn xrefX
% definition. WARNING: may need to track texinfo.tex changes to xrefX...
\newif\ifbra\bratrue
\def\altref{\ifbra\brafalse\else\bratrue\fi}% Toggle.
%%%%%%%%%%%%%%%%%%%%%%%%%%%%%%Begin duplicate xrefX:
\def\xrefX[#1,#2,#3,#4,#5,#6]{\begingroup%
\def\printedmanual{\ignorespaces #5}%
\def\printednodename{\ignorespaces #3}%
%
\setbox1=\hbox{\printedmanual}%
\setbox0=\hbox{\printednodename}%
\ifdim \wd0=0pt%
\def\printednodename{\ignorespaces #1}%
%%% Uncommment the following line to make the actual chapter or section title
%%% appear inside the square brackets.
%\def\printednodename{#1-title}%
\fi%
%
%
% If we use \unhbox0 and \unhbox1 to print the node names, TeX does
% not insert empty discretionaries after hyphens, which means that it
% will not find a line break at a hyphen in a node names.  Since some
% manuals are best written with fairly long node names, containing
% hyphens, this is a loss.  Therefore, we simply give the text of
% the node name again, so it is as if TeX is seeing it for the first
% time.
\ifdim \wd1>0pt
section ``\printednodename'' in \cite{\printedmanual}%
\else%
\turnoffactive%
%%%%%%%%%%%%%%%Here's the change for @altref:
\ifbra
\refx{#1-snt}{} ``\printednodename,'' page\tie\refx{#1-pg}{}%
\else
\refx{#1-snt}{} \printednodename, page\tie\refx{#1-pg}{}%
\fi
%%%%%%%%%%%%%%%end change for @altref
\fi
\endgroup}
%%%%%%%%%%%%%%%%%%%%%%%%%%%%%%%:End duplicate xrefX
%%%%%%%%%%%%%%%%%%%%%%%%%%%%%%%%%%%%%%%%%%%%%%%%%%%%%%%%%%%%
% \widen: let examples (or whatever) fall into margins.  Symmetric.  Cancel
% ``@widen{N}'' with ``@widen{-N}''.
\newdimen\extra\extra=0pt % for ``widen''
\global\def\widen#1{\advance\extra by #1%
\advance\leftskip by -#1\advance\rightskip by -#1}
%
%%%%%%%%%%%%%%%%%%%%%%%%%%%%%%%%%%%%%%%%%%%%%%%%%%%%%%%%%%%%
% ONLY INTENDED CHANGES from FSF smallbook: tolerance and page centering
% Use @smallbook to reset parameters for 7x9.5 format  (or else 7x9.25)
\def\smallbook{

% These values for secheadingskip and subsecheadingskip are
% experiments.  RJC 7 Aug 1992
\global\secheadingskip = 17pt plus 6pt minus 3pt
\global\subsecheadingskip = 14pt plus 6pt minus 3pt

\global\lispnarrowing = 0.3in
\setleading{12pt}
\advance\topskip by -1cm
\global\parskip 3pt plus 1pt
\global\hsize = 5in
\advance\normaloffset by .75in % half of difference from 6.5in default hsize
\global\vsize=7.5in
\advance\voffset by .7in % half of difference from 8.9in default vsize
\global\tolerance=1400
\global\hfuzz=1pt
\global\contentsrightmargin=0pt

\global\pagewidth=\hsize
\global\pageheight=\vsize

\global\let\smalllisp=\smalllispx
\global\let\smallexample=\smalllispx
\global\def\Esmallexample{\Esmalllisp}
}\textfonts

\globaldefs=0 % this is NOT redundant; the \endgroup done by@end tex
	      % would restore the value, but before doing that would
	      % screw up if globaldefs=1 were allowed to remain here.
@end tex
@rm
@smallbook
@c some manuals (notably as) contain stuff that only looks good in
@c smallbook; this switch controls its appearance, by local convention
@set SMALL
